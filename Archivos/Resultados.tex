\chapter{Resultados}

\section{Entorno Experimental}



\subsection{robot}
La plataforma que estamos usando es un robot cartesiano de 3 DOF visto en la Fig. \ref{fig:dsc9825}, se utiliza un generador de señal para el voltaje de entrada a los motores de corriente continua, estamos utilizando MATLAB como la interfaz entre el sistema de visión, el control y el generador de funciones.

\begin{figure}
	\centering
	\includegraphics[width=0.5\linewidth]{visio/visio3/coordenadasrobcam2}
	\caption{}
	\label{fig:coordenadasrobcam}
\end{figure}


\begin{figure}
	\centering
	\includegraphics[width=0.5\linewidth]{Imagenes/DSC_9825}
	\caption{}
	\label{fig:dsc9825}
\end{figure}



\subsection{\textit{Gripper}}
El \textit{Gripper} que vamos a utilizar es el modelo SCHUNK WSG-50 que aparece en \cref{fig:dsc9821}, tiene un observador de fuerza, utiliza una fuente de 24V, la abertura es de 10 cm y se puede utilizar con MATLAB para ser operado. \\
Para ser usado se necesitó establecer la comunicación con \textbf{MATLAB}, esta es comunicación básica, en la que se manda un paquete de datos que contiene el comando y las opciones, en el apéndice \ref{codewsg50}, aparece el código que se uso.

\begin{figure}
	\centering
	\includegraphics[width=0.7\linewidth]{Imagenes/DSC_9821}
	\caption{}
	\label{fig:dsc9821}
\end{figure}

\subsubsection{base del \textit{Gripper}}
Se diseño la base para el \textit{Gripper}, y los dedos, se intenta usar el sensor de fuerza y momento, por lo que uno de los 2 dedos tiene una cavidad para este. en el apéndice \ref{label}, se puede encontrar el diseño que se uso.

en la imagen \ref{label}, se encuentra el resultado




\subsection{cámara}

El sistema de visión usa una cámara RGBD que se muestra en la figura \ref{fig:dsc9822}, se uso un programa entre MATLAB y Openni \cite{matlabwrapper}.

La cámara que se uso es el ASUS XTION PRO, su sensor de profundidad tiene un rango operativo de 0,8 metros a 3,5 metros, la resolución de la imagen de color y la imagen de profundidad es $480 \times640$, y la velocidad de fotogramas es 20 cuadros por segundo, ya que es la velocidad de fotogramas del sensor de profundidad. \\

esta cámara trabaja con el software Openni, que es la librería en c++, esta librería tiene funciones para usar la cámara y hacer ciertas operaciones,

al principio se pensaba usar la imagen de rango del sensor de profundidad, que es la distancia desde el sensor hasta el punto censado, esto se pudo evitar gracias a las funciones integradas en Openni.

MATLAB, en ninguna de sus versiones tiene soporte oficial para las cámaras ASUS, aunque si tiene soporte para nuevos dispositivos, pero existe el programa envolvente entre \textbf{MATLAB} y Openni.
el problema con este programa es que requiere de unas condiciones muy especificas para poder ser instalado. entre esas condiciones esta

-usar \textbf{MATLAB} 2010
-compilar con \textbf{Microsoft visual Studio} 2010


una vez instalado se puede usar en otras versiones de \textbf{MATLAB}, en este caso se uso en la versión 2015


\begin{figure}
	\centering
	\includegraphics[width=0.7\linewidth]{Imagenes/DSC_9823}
	\caption{}
	\label{fig:dsc9823}
\end{figure}
\begin{figure}
	\centering
	\includegraphics[width=0.7\linewidth]{Imagenes/DSC_9824}
	\caption{}
	\label{fig:dsc9824}
\end{figure}


\section{Protocolo de Experimentación}

\section{Conclusiones}


