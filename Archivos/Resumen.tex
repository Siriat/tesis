\chapter*{\centerline{\Huge\bf Resumen}}
\addcontentsline{toc}{chapter}{Resumen}
\vspace{1em}
\begin{center}{\Large\bf {\titlenamesp}}\end{center}
\vspace{3em}
Sujetar objetos es la tarea mas común para un robot inercial, en esta tarea se necesita una buena velocidad y precisión pero un bajo impacto en el contacto, en este trabajo se implementara un control de redes neurodifusas que tienen como entrada las distancias [x,y,z] entre desde el gripper hasta el objeto, usando una cama fija rgbd, ademas de usar sensores para medir la posición real y la fuerza del gripper, esto para comparar los resultados.
%Lo que se espera es ver el desempeño de las redes neurodifusas en esta tarea, se espera que las redes eliminen los errores que podría haber debido a una mala identificación de la cámara, mientras el error no sea muy grande, también se espera que la fase de sujetar sea lenta, debido a la lógica difusa.
Los experimentos se llevaran acabo con un robot cartesiano, en condiciones controladas, para tomar un objeto cuya posición es desconocida.

% Se recomienda sólo una cuartilla
%%%%%%%%%%%%%%%%%%%%%%%%%%%%%%%%%%%%%%%%%%%%%%%%%%%%%%%%%%%%%%%%%%%%%%%%%%%%%%%%%%%%%%%%%%%%%%%%%%%%




%%%%%%%%%%%%%%%%%%%%%%%%%%%%%%%%%%%%%%%%%%%%%%%%%%%%%%%%%%%%%%%%%%%%%%%%%%%%%%%%%%%%%%%%%%%%%%%%%%%%
\chapter*{\centerline{\Huge\bf Abstract}}
\addcontentsline{toc}{chapter}{Abstract}
\vspace{1em}
\begin{center}{\Large\bf {\titlenameen}}\end{center}
\vspace{3em}
%Grasping is one of the most common task for a robot, but it requires specific knowledge about the object and the environment.
%Humans recognize the object, remember the proprieties it had (assume), and grasp it, if they don’t know the object they still can guess some proprieties and try to grasp it, and then correct the force, position, speed, grasping point based on their previous knowledge.
%Working on unstructured places is difficult due to the lack of information, but lately the flexibility of the system is becoming more important; to deal with uncertainties or changing proprieties. A control scheme is proposed for the grasping task of an unknown object, using a vision system and a force sensor, the vision system is composed of a RGBD camera, for the positioning of the robot, and an ultrasonic sensor which, should be able to detect the distance to the object for the grasping part, and the recognition of the object material, restricted to solid objects.

% Se recomienda sólo una cuartilla
%%%%%%%%%%%%%%%%%%%%%%%%%%%%%%%%%%%%%%%%%%%%%%%%%%%%%%%%%%%%%%%%%%%%%%%%%%%%%%%%%%%%%%%%%%%%%%%%%%%%




%%%%%%%%%%%%%%%%%%%%%%%%%%%%%%%%%%%%%%%%%%%%%%%%%%%%%%%%%%%%%%%%%%%%%%%%%%%%%%%%%%%%%%%%%%%%%%%%%%%%
