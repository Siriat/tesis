\chapter{Conclusiones}
%As there can be seen the decision of using FREN is still to early to be set, but the adaptation part is still desirable, the vision system can use the deep part to track the object once it is found, as there is no longer need to look for it, taking in count it does not encounters another object on the way, with similar shape.\\
  %  \section{Future work}
%The ultrasonic sensor has not been used, this would be the next step of the work making the gasping easier as the vision system is not able to accurately measure the distance between the \textit{Gripper} fingers and the object, this would solve Fig. 1.  the problem of the force sensor (the force grows too fast to be controlled at certain speeds).\\
%There should be a sujeción point algorithm implemented, the experiments would be more significant in this way.\\






%What did you do.....and what do you have....How (technique, algorithm)

%In this work, the image processing technique has been proposed to identify the object with 2.5D reconstruction. 
%The dimension and position of the object have been obtained.
%Thus, the control algorithm based on FREN has been implemented in the Cartesian robot to move the \textit{Gripper}.
%In this work, the image processing technique has been proposed to identify the object with 2.5D reconstruction. The dimension and position of the object have been obtained. Thus, the control algorithm based on FREN has been implemented in the Cartesian robot to move the \textit{Gripper}. %How it's going on....

En este trabajo se ha propuesto la técnica de procesamiento de imágenes para identificar el objeto con reconstrucción 2.5D. Se han obtenido la dimensión y posición del objeto. Por lo tanto, el algoritmo de control basado en FREN se ha implementado en el robot cartesiano para mover la pinza.


%The experimental results of 2.5D reconstruction has demonstrated that if the object is symmetric, an incomplete model of the object is enough to determine the centroid, but the algorithm for sujeción is infective on some shapes.
%The experimental results of 2.5D reconstruction has demonstrated that if the object is symmetric, an incomplete model of the object is enough to determine the centroid.
Los resultados experimentales de la reconstrucción 2.5D ha demostrado que si el objeto es simétrico, un modelo incompleto del objeto es suficiente para determinar el centroide.

%, yet the algorithm may present difficulties for sujeción some shapes because there was not a propper sujeción point algorithm. %the main problem with the experiments would be the slow processing of the segmentation, and the 
%Controller:
%The control performance proved to work stable in reaching the object, while the implementation was really simple.
El rendimiento de control resultó ser estable en alcanzar el objeto, mientras que la implementación fue realmente simple.

%Future work .....      


%Some future work would be the implementation of a algorithm for a proper sujeción point finding based on the normals of the surface to find the place with more sujeción area, also the tracking of the \textit{Gripper} and the object, 
%Some future work would be the implementation of an algorithm for a proper sujeción point finding based on the normals of the surface to find the place with more sujeción area, furthermore an image registration could not be implemented which could have related color and shape features to help the tracking of the \textit{Gripper} and the object, and the use of a FREN controller for force regulation.
Algunos trabajos futuros serían la implementación de un algoritmo.
Para un punto apropiado de agarre que encuentra basado en las normales De la superficie para encontrar el lugar con más área de agarre, además no se pudo implementar un registro de imagen que pudiera tener características de color y forma relacionadas para ayudar al seguimiento de la pinza y el objeto, y el uso de una Controlador FREN para la regulación de la fuerza.

%which presented difficulties because of the computational cost. checar!!!!!!!!!!!
%and the use of a FREN controller for force regulation.


\section{discusiones}



En el caso del sujeción es donde hay mas problemas, siendo que no se tiene una métrica adecuada para poder decidir la fuerza necesaria para levantar el objeto.

Uno de los problemas principales es el \textit{Gripper} en si, en el caso de tener ordenes concisas, el \textit{Gripper} probo un buen desempeño, pero en este caso las ordenes están sujetas a cambios dependiendo de la información recidiva por los sensores.
Otro problema importante es la comunicación con el \textit{Gripper}, la cual oscila de entre 0.5-1.5 seg, esta velocidad de respuesta es muy lenta para la aplicación de ajuste de fuerza de sujeción. en la comunicación existen otros problemas como lo es la dificultad para detener un proceso y cambiarlo por otro.

Por estas razones e sugiere el uso de un \textit{Gripper} que pueda ser controlado a bajo nivel.

Una de las cosas que no se probo fue el hecho de que el \textit{Gripper} es capa de correr programas en su interior y responder acorde con el código, desde el interior del \textit{Gripper} se puede realizar la lectura de los sensores que estén en los dedos del \textit{Gripper}, y en base a esto reaccionar según la situación, pero de este enfoque se esperan algunos problemas como:
1) No se tienen los dedos adecuados para ser conectados.
2) No se sabe la velocidad que se llegara a tener en la lectura.
3) No se sabe la velocidad de procesamiento que se puede tener.
4) No se sabe si la velocidad de respuesta aumentara o seguirá igual.







Algunos de los métodos que se siguieren son:



El uso de una matriz de sensores de presión

Debido a las condiciones que se tienen en el sistema esta podría ser una buena solución.

En este robot el \textit{Gripper} siempre tomara los objetos desde la parte de arriba,  así que usando una matriz de sensores se vería como en la imagen \ref{fig:matrix}:

\begin{figure}
	\centering
	\includegraphics[width=0.4\linewidth]{visio/matrix}
	\caption{}
	\label{fig:matrix}
\end{figure}

Esto se debe a que solo se puede tomar desde la parte de arriba y siempre debería haber un espacio entre el \textit{Gripper} y el objeto, con esto podemos tener un indicador de la posición del objeto con respecto al \textit{Gripper}, la matriz puede ser vista como una imagen de presión por lo que se puede conseguir un centroide, cuando este centroide, en el momento en que se deslice, sera hacia abajo, disminuyendo la presión ejercida en cada una de las celdas del borde baje y el centroide también cambiara de posición.

El cambio del centroide en esta imagen de presión es un buen indicador de que existe deslizamiento, ademas de ayudar a indicar cual es la velocidad a la que se desliza, obviamente se sugiere uno en cada dedo, algunas de las ventajas son la velocidades de lectura de los sensores.
Una de las consideraciones que deberían tenerse al hacer esto es que es recomendable tener una capa de material deformable entre el sensor y el objeto. Esto tiene 2 razones, el agarre por fuerza solo es valido con 3 puntos de contacto no co-lineales o 2 puntos, para un materia deformable, el segundo es que al momento de hacer presión sobre el sensor, no es seguro que el contacto con el sensor sea bueno, al tenerlo recubierto de un material deformable, la presión se distribuirá mejor cuando llegue al sensor.


Otra de las propuestas que se tienen es el uso de 2 sensores de \textit{Wrench}.

En vista que se busca crear una base de datos de manera automática para el aprendizaje del robot, me parece que seria de gran ayuda conocer el centro de masas real del objeto, por lo que se propone el uso de 2 sensores de \textit{Wrench} en los dedos del \textit{Gripper}.

Se piensa que al momento de sostener un objeto, el centro de masas creara torques en respuesta a los momentos que se tengan, si se tuviera solo uno, los momentos que se dirigen al otro dedo se perderían.



El uso de programación inductiva para la base de datos inteligente.

La programación inductiva es un método muy distinto a la lógica difusa y las redes neuronales en el hecho de que es completamente discreto y usa solo relaciones lógicas en lugar de operaciones matemáticas.

Uno de los trabajos futuros que se planea tener es generar una base de datos para guardar los objetos que ya se encontraron, y tratar los objetos parecidos como objetos familiares, el solo hecho de guardar la información que se consigue no es suficiente, por lo que se propone usar programación inductiva para poder hacer una relación entre los objetos que se encuentran y poder hacer hipótesis hacerla de lo probable que seria que el objeto encontrado sea igual a otro objeto ya encontrado.

Esto se hace construyendo una tabla de datos para cada encuentro de un nuevo objeto, y usando reglas de inducción, para ver la relación que hay entre que los objetos tengan un tamaño y su peso.



