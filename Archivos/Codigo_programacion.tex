\definecolor{gray97}{gray}{.97}
\definecolor{bluekeywords}{rgb}{0.13,0.13,1}
\definecolor{greencomments}{rgb}{0,0.5,0}
\definecolor{redstrings}{rgb}{0.9,0,0}
%\DeclareCaptionFormat{listing}{\colorbox{gray}{\parbox{\textwidth}{#1#2#3}}}
%\captionsetup[lstlisting]{format=listing,labelfont=white,textfont=white}

\lstset{language=Matlab, %c#  ó c++
showspaces=false,
showtabs=false,
breaklines=true,
showstringspaces=false,
breakatwhitespace=true,
escapeinside={(*@}{@*)},
keywordstyle=\color{bluekeywords}\bfseries,
backgroundcolor=\color{gray97},
%%%
numberstyle=\footnotesize,
numbers=left,
stepnumber=1,
numbersep=10pt,
tabsize=2,
breaklines=true,
prebreak = \raisebox{0ex}[0ex][0ex]{\ensuremath{\hookleftarrow}},
breakatwhitespace=false,
aboveskip={1.5\baselineskip},
columns=fixed,
upquote=true,
extendedchars=true,
frame=lines,
}
\chapter{Código de programación}\label{codigo}

Ejemplo de un código de programación:

\section{Anexo I. Código de visión}\label{visioncode}
 \begin{lstlisting}
 D=permute(D,[2 1 3]);
 tic
 coor(:,1)=reshape(D(:,:,1),[sx*sy,1]);
 coor(:,2)=reshape(D(:,:,2),[sx*sy,1]);
 coor(:,3)=reshape(D(:,:,3),[sx*sy,1]);
 newcoor=coor*R;
 x=reshape(newcoor(:,1),[sy,sx])+dx;
 y=reshape(newcoor(:,2),[sy,sx])+dy;
 z=reshape(newcoor(:,3),[sy,sx])+dz;
 
 y(x>maxX)=nan;
 z(x>maxX)=nan;
 x(x>maxX)=nan;
 
 x(or(y<maxY,y>minY))=nan;
 z(or(y<maxY,y>minY))=nan;
 y(or(y<maxY,y>minY))=nan;
 
 x(z>maxZ)=nan;
 y(z>maxZ)=nan;
 z(z>maxZ)=nan;
 
 xyz=cat(3,x,y,z);
 %figure;mesh(xyz(:,:,1),xyz(:,:,3),xyz(:,:,2))
 %axis equal
 
 
 
 mx=mean(x(dix~=0));
 my=mean(y(diy~=0));
 mz=mean(z(diz~=0));
 %despues se consigue el error con respecto de x
 erx=abs(x-mx);
 raycasting=z((erx<5));
 %pz(l)=max(raycasting);
 %py(l)=my;
 %px(l)=mx;
 %raycastedy=y((erx<5));
 %wi(l)=max(raycastedy)-min(raycastedy);
 
 t(l)=toc;
 end
 
 dpos=[px py pz];
 \end{lstlisting}

\section{Anexo II. Código de FREN}\label{FRENcode}

\subsection{Evaluación de FREN}
\begin{lstlisting}
function [lc, per]=fren(member,ship)
type=ship.type;
class=ship.class;
c=ship.c;
b=ship.b;
betas=ship.betas;
f=ship.f;
%per=zeros(1,l);
switch class
case 'fren'
l=ship.l;

switch type
case 'tri'
pass=ship.pass;
for i=1:l
for j=1:3
if pass{i}{j}(member,c(i),b(i))
per(i)=f{i}{j}(member);
end
end
end
case 'gauss'
for i=1:l
per(i)=f{i}{1}(member,c(i),b(i));
end

end
lc=per*betas';

case 'frend'
index=ship.index;
switch type

case 'gauss'
for j=1:index
for i=1:length(c{j})
per{j}(i)=f{j}{i}{1}(member(j),c{j}(i),b{j}(i));
end 
per{j}=per{j}./sum(per{j});
end


if mod(index,2)==1
ain=index;
bin=index-1;
else
ain=index-1;
bin=index;
end
[a,~]=rekron(per,ain,'evaluation');
[b,~]=rekron(per,bin,'evaluation');
%   lc=per*betas'*perd*betasd';  %check
lc=a*betas*b';
end
end

end
\end{lstlisting}


\subsection{Creación de una FREN}
\begin{lstlisting}
function ship=shipconstruction(class,type,center,base,betas,y,eta)

switch class

case 'fren'
l=length(center);
ship=struct('class',class,'type',type,'c',center,'b',base,'l',l,'betas',betas,'y',y,'eta',eta);

syms mem
switch type
case 'tri'
passi={@(mem,c,b)(mem<=c),@(mem,c,b)(mem>=c)&&(mem<=(c+b)),@(mem,c,b)(mem>=(c+b))};
passm={@(mem,c,b)(mem<=c)&&(mem>=c-b),@(mem,c,b)(mem>=c)&&(mem<=c+b),@(mem,c,b)(mem<=c-b)||(mem>=c+b)};
passf={@(mem,c,b)(mem>=c),@(mem,c,b)(mem<=c)&&(mem>=c-b),@(mem,c,b)(mem<=c-b)};

ship.pass{1}=passi;
f=(-1/base(1))*mem+(center(1)/(base(1))+1);
f=cat(2,'@(mem)',char(f));
a=str2func(f);
ship.f{1}={@(mem)1,a,@(mem)0};
for i=2:l-1
ship.pass{i}=passm;
f=(1/base(i))*(mem-center(i))+1;
f=cat(2,'@(mem)',char(f));
a=str2func(f);
f=(-1/base(i))*mem+(center(i)/(base(i))+1);
f=cat(2,'@(mem)',char(f));
b=str2func(f);
ship.f{i}={a,b,@(mem)0};
end
ship.pass{l}=passf;
f=(1/base(1))*mem+(center(1)/(base(1))+1);
f=cat(2,'@(mem)',char(f));
a=str2func(f);
ship.f{l}={@(mem)1,a,@(mem)0};

case '+-'

case 'gauss'
syms In a b
Mu=(1./(1+exp(2*pi*(1/b)*(In-a-b/2))));
f=cat(2,'@(In,a,b)',char(Mu));
f=str2func(f);
ship.f{1}={f};
for i=2:l-1
f=exp(-((In-a)^2)/(b^2/pi));
f=cat(2,'@(In,a,b)',char(f));
f=str2func(f);
ship.f{i}={f};
end
Mu=(1./(1+exp(-2*pi*(1/b)*(In-a+b/2))));
f=cat(2,'@(In,a,b)',char(Mu));
f=str2func(f);
ship.f{l}={f};

case 'tan'

end

=========================== MiFREN=============================
case 'frend'
index=length(center);
ship=struct('class',class,'type',type,'index',index,'y',y,'eta',eta);
ship.c=center;
ship.b=base;
%        ship.betas=betas;
[ship.betas,~]=rekron(betas,index,'construction');
switch type
case 'gauss'

syms In a b
for j=1:index
l=length(center{j});
Mu=(1./(1+exp(2*pi*(1/b)*(In-a-b/2))));
f=cat(2,'@(In,a,b)',char(Mu));
f=str2func(f);
ship.f{j}{1}={f};
for i=2:length(center{j})-1
f=exp(-((In-a)^2)/(b^2/pi));
f=cat(2,'@(In,a,b)',char(f));
f=str2func(f);
ship.f{j}{i}={f};
end
Mu=(1./(1+exp(-2*pi*(1/b)*(In-a+b/2))));
f=cat(2,'@(In,a,b)',char(Mu));
f=str2func(f);
ship.f{j}{l}={f};
end
end
end
\end{lstlisting}

\subsection{fase de aprendizaje}
\begin{lstlisting}
function ship=shipmodernization(ship,per,e)
y=ship.y;
betas=ship.betas;
eta=ship.eta;

class=ship.class;
switch class
case 'fren'
ship.betas=betas+eta*y*e*per;
case 'frend'
ship.betas=betas+eta*y*e*per;
end
\end{lstlisting}

\subsection{Programa de recursivo para producto \textbf{Kroneker}}
\begin{lstlisting}
function [w,index]=rekron(crew,index,type)
a=crew{index};

switch type
case 'construction'
if index==2
w=kron(a,crew{1}');
elseif mod(index,2)==1
w=kron(a,rekron(crew,index-1,'construction')')';
elseif mod(index,2)==0
w=kron(a,rekron(crew,index-1,'construction'));
end

case 'evaluation'
if index<=2
w=a;
elseif index>2
w=kron(a,rekron(crew,index-2,'evaluation'));
end

end
\end{lstlisting}


\section{Anexo III. Comunicación con WSG50}\label{codewsg50}
\begin{lstlisting}
%comunication
%obj = tcpip('192.168.1.20', 1000, 'NetworkRole', 'client');
%fopen(obj)
function [id, statuscode, payload] =wsg50(obj,cmd,stat)
%==============//////////note: falta check sum////////==================
%command
%check database, this migth take some time
% id='a';
% statuscode='b'; 
% payload='c';
if nargin ==3
stat=single(stat);
end
switch cmd

case 'connect'
%   obj = tcpip('192.168.1.20', 1000, 'NetworkRole', 'client');
obj.ByteOrder = 'littleEndian';
fopen(obj);
return

case 'disconnect'
%   cmd='07';
pack={ 'AA', 'AA', 'AA', '07', '00', '00',  '35', '4C'};
fwrite(obj,hex2dec(pack),'char')
fclose(obj);
id=0;
return
case 'homing'
%   cmd='20';
pack={ 'AA', 'AA', 'AA', '20', '01', '00', dec2hex(stat), '00', '00'};

case 'preposition'
%   cmd='21';
v=num2hex([stat(2),stat(3)]);
pack={ 'AA', 'AA', 'AA', '21', '09', '00', dec2hex(stat(1)),v(1,7:8),v(1,5:6),v(1,3:4),v(1,1:2),v(2,7:8),v(2,5:6),v(2,3:4),v(2,1:2), '00', '00'};

case 'stop'
%   cmd='22';
pack={ 'AA', 'AA', 'AA', '22', '00', '00', '00', '00'};

case 'faststop'
%   cmd='23';
pack={ 'AA', 'AA', 'AA', '23', '00', '00', '00', '00'};

case 'aknwfaststop'
%   cmd='24';
pack={ 'AA', 'AA', 'AA', '24', '03', '00', '61', '63', '6B' , '00', '00'};

case 'grip'
%   cmd='25';
v=num2hex([stat(1),stat(2)]);
pack={ 'AA', 'AA', 'AA', '25', '08', '00',v(1,7:8),v(1,5:6),v(1,3:4),v(1,1:2),v(2,7:8),v(2,5:6),v(2,3:4),v(2,1:2), '00', '00'};

case 'release'
%   cmd='26';
v=num2hex([stat(1),stat(2)]);
pack={ 'AA', 'AA', 'AA', '26', '08', '00',v(1,7:8),v(1,5:6),v(1,3:4),v(1,1:2),v(2,7:8),v(2,5:6),v(2,3:4),v(2,1:2), '00', '00'};

case 'dforce'
%   cmd='32';
v=num2hex(stat);
pack={ 'AA', 'AA', 'AA', '32', '04', '00',v(1,7:8),v(1,5:6),v(1,3:4),v(1,1:2), '00', '00'};

case 'opening?'
%   cmd='43';
v=dec2hex(cat(2,stat,4096));
pack={ 'AA', 'AA', 'AA', '43', '03', '00',v(1,3:4),v(2,3:4),v(2,1:2), '00', '00'};

case 'speed?'
%   cmd='44';
v=dec2hex(cat(2,stat,4096));
pack={ 'AA', 'AA', 'AA', '44', '03', '00',v(1,3:4),v(2,3:4),v(2,1:2), '00', '00'};

case 'force?'
%   cmd='45';
v=dec2hex(cat(2,stat,4096));
pack={ 'AA', 'AA', 'AA', '45', '03', '00',v(1,3:4),v(2,3:4),v(2,1:2), '00', '00'};

case 'read'
[id, statuscode, payload] = read_data(obj);
return

otherwise
disp('command not implemented')
return
end

fwrite(obj,hex2dec(pack),'char')
% estaparte deberia ser usada aparte, despues del  comando, o como cotro
% comando
% disp('1');
[id, statuscode, payload] = read_data(obj); %wait for acknowledgement
% disp('2');
%  [id(:,:,2), statuscode(:,:,2), payload(:,:,2)] = read_data(obj); %wait for command to be executed


function [command_id, statuscode, payload] = read_data(t)
%wait for preamble
command_id=nan;
statuscode=nan;
payload=nan;

while (fread (t,1) ~= 170)  
end;
fread (t,2);  %read the rest of the preamble
%read the command ID
command_id = fread (t,1);
%read the length of the payload (two bytes)
length= fread (t,1,'uint16');
%read the status code (two bytes)
statuscode = fread(t, 1, 'uint16');

switch statuscode
case 0
statuscode='success';
case 1
statuscode='data not avaible';
case 2
statuscode='no sensor';
case 3
statuscode='not initialized';
case 4
statuscode='already running';
case 5
statuscode='not suported';
case 6
statuscode='inconsistent data';
case 7
statuscode='timeout';
case 8
statuscode='read error';
case 9
statuscode='write error';
case 10
statuscode='low memory';
case 11
statuscode='checksum';
case 12
statuscode='no parameter expected';
case 13
statuscode='not enough parameters';
case 14
statuscode='unkwn cmd';
case 15
statuscode='format error';
case 16
statuscode='access denied';
case 17
statuscode='already open';
case 18
statuscode='error while excecuting the comand';
case 19
statuscode='cmd aborted';
case 20
statuscode='invalid ahndle';
case 21
statuscode='device or file not found';
case 22
statuscode='device or file not open';
case 23
statuscode='i/o error';
case 24
statuscode='wrong parameter';
case 25
statuscode='index out of bounds';
case 26
statuscode='cmd pending';
case 27
statuscode='data overrun';
case 28
statuscode='range error';
case 29
statuscode='axis blocked';
case 30
statuscode='file already exist';
end


if strcmp(statuscode,'success')
length = length - 2;
else
length=0;
return
end

%decrease the length of the payload by the statuscode bytes

%read the payload
if ( length > 0 )
payload = fread(t, length);
payload = double(typecast(uint8(payload), 'single'));
else
payload = 0 ;
end
%read the checksum
% checksum = fread (t, 1,'uint16');
\end{lstlisting}


\section{Anexo IV. Código Completo}\label{codefull}
\begin{lstlisting}
% robot 1.8
%------change log--------------------------------------------------------------
% 1.6   este tiene la funcion freen.m
% 1.7   aqui se agrega la parte 2.5D  
% 1.8   este es cuando se hace en lazo abierto.
%---------------------------------------------------------------------
for l=1:1
clear;
%% -------  CONTROL --------
center=[-5 -2 0 2 5];
base=[3 2 2 2 3];
betas=[-5000 -500 1e-6 500 5000];
% shipx=shipconstruction('fren','gauss',center,base,betas,1,1);
shipx=shipconstruction('gauss',center,base,betas,1,1);

shipy=shipx;
shipz=shipx;

%% vision
%%%antes
load gripperfeatures.mat
close all;
addpath('Mex')
SAMPLE_XML_PATH='Config/SamplesConfig.xml';
KinectHandles=mxNiCreateContext(SAMPLE_XML_PATH);
%%% variables
iter=5;
tolerance=.3;
inicial=41;
final=440;
totaly=final-inicial;
gripperarea=4700; %esto tal vez no sea necesario
trigger=0;
target='lost';
delta=.3;
flagj=0;
flagi=0;
minZ=700;
maxZ=1000;
object=struct('cent',[0,0],'area',0,'box',[0 0 0 0],'x',0,'y',0);
gripper=struct('cent',[0,0],'area',0,'box',[0 0 0 0],'x',0,'y',0);
se90 = strel('line', 5, 90);
se0 = strel('line',5, 0);


%% INIT-INSTRUMENTS
obj2 = instrfind('Type', 'visa-gpib', 'RsrcName', 'GPIB0::10::0::INSTR', 'Tag', '');

% Create the VISA-GPIB object if it does not exist
% otherwise use the object that was found.
if isempty(obj2)
obj2 = visa('NI', 'GPIB0::10::0::INSTR');
else
fclose(obj2);
obj2 = obj2(1)
end

% Connect to instrument object, obj2
set (obj2,'OutputBufferSize',100000);
fopen(obj2);

fprintf (obj2, '*IDN?');
idn = fscanf (obj2);
fprintf (idn)
fprintf ('\n\n')

%Clear and reset instrument
fprintf (obj2, '*RST');
fprintf (obj2, '*CLS');

%Set desired configuration.
fprintf(obj2,'FUNCTION SQU');
fprintf(obj2,'VOLT 2.5'); % set max waveform amplitude to 2 Vpp
fprintf(obj2,'VOLT:OFFSET 1.25'); % set offset to 0 V
fprintf(obj2,'OUTPUT:LOAD 50'); % set output load to 50 ohms


% System input/output definiftion--------------

do=digitalio('nidaq','Dev1');     % Data to move motors 
addline(do,0:5,'Out');

DIRX=1;
DIRY=2;
DIRZ=3;


MX=4;
MY=5;
MZ=6;

RIGHT=1;
LEFT=0;

ON=1; 
OFF=0;

% Add initial for function GEN 
putvalue(do.line(MX),ON);
putvalue(do.line(MY),ON);
putvalue(do.line(MZ),OFF);
% -------------  

%% ---------Constants for position----------
kmax=1000;
u_max=20000;
u=zeros(1,kmax+1);

xd=u;%preallocation, en caso de que se quiera seguir una trayectoria, si se quiere constante, borrar xd"(k+1)"
x=u;
y=u;
z=u;
e=zeros(kmax+1,3);
eo=e;
Ts=u;
R_Time=u;
km=6.0049;
kb=-0.054;

%---Voltage reading 
aiv=analoginput('nidaq','Dev1')
addchannel(aiv,0)
set(aiv,'InputType','SingleEnded')
V_z=getsample(aiv);    % Read the voltage ... 
PosZ=km*V_z+kb;
x(1)=PosZ(1);  % Psition  [cm]






%% Control Force Loop

% ------ Turn ON Function Gen -----
fprintf(obj2,'OUTPUT ON');
F_END=0;
fprintf(obj2,['FREQ ', num2str(F_END)]); %set frequency to Hz
end

% while resp==r
% i=1;
% ra=10;
% s.BaudRate=ra*i;
% fopen(s);
% resp=fscanf(s)
% fclose(s);
% if i>1000
%     break
% end
% end

%------------------ parte 2.5D ------------------

mxNiUpdateContext(KinectHandles);
D=mxNiDepthRealWorld(KinectHandles);

sx=640;
sy=480;

maxX=1600;
maxY=-1600;
maxZ=1450;
minY=1000;
dx=-500;
dy=1000;
dz=200;


al=(0)*pi/180;
be=(0)*pi/180;
ga=(0)*pi/180;

rx=[1 0 0;0 cos(al) -sin(al);0 sin(al) cos(al)];
ry=[cos(be) 0 sin(be);0 1 0;-sin(be) 0 cos(be)];
rz=[cos(ga) -sin(ga) 0;sin(ga) cos(ga) 0;0 0 1];
R=rx*ry*rz;

for l=1:1

D=permute(D,[2 1 3]);
tic
coor(:,1)=reshape(D(:,:,1),[sx*sy,1]);
coor(:,2)=reshape(D(:,:,2),[sx*sy,1]);
coor(:,3)=reshape(D(:,:,3),[sx*sy,1]);
newcoor=coor*R;
x=reshape(newcoor(:,1),[sy,sx])+dx;
y=reshape(newcoor(:,2),[sy,sx])+dy;
z=reshape(newcoor(:,3),[sy,sx])+dz;

y(x>maxX)=nan;
z(x>maxX)=nan;
x(x>maxX)=nan;

x(or(y<maxY,y>minY))=nan;
z(or(y<maxY,y>minY))=nan;
y(or(y<maxY,y>minY))=nan;

x(z>maxZ)=nan;
y(z>maxZ)=nan;
z(z>maxZ)=nan;

xyz=cat(3,x,y,z);
%figure;mesh(xyz(:,:,1),xyz(:,:,3),xyz(:,:,2))
%axis equal



mx=mean(x(dix~=0));
my=mean(y(diy~=0));
mz=mean(z(diz~=0));
%despues se consigue el error con respecto de x
erx=abs(x-mx);
raycasting=z((erx<5));
%pz(l)=max(raycasting);
%py(l)=my;
%px(l)=mx;
%raycastedy=y((erx<5));
%wi(l)=max(raycastedy)-min(raycastedy);

t(l)=toc;
end

dpos=[px py pz];
%------------------- Main Loop ----------------
R_Time(1)=0;
for k=1:kmax-1 
%% position
V_x=getsample(aiv);    % Read the voltage ...
pos=km*V_x+kb;


%% freen control
eu=dpos-pos; %e(iter,:)%mean(e(k-2:k,:))
if  abs(eu(1))>10
putvalue(do.line(MX),ON);
putvalue(do.line(MY),OFF);
putvalue(do.line(MZ),OFF);
[u_tempx, perx]=fren(eu(1),shipx);
% display(eu(1));
display(u_tempx);

if abs(u_tempx)>u_max
u(k)=sign(u_tempx)*u_max;
else
u(k)=u_tempx;
end1

%--- Send control effort to system ---
if u(k)>=0
putvalue(do.line(DIRX),RIGHT);  % Down --
else
putvalue(do.line(DIRX),LEFT);  % Up --
end
F=abs(u(k));
fprintf(obj2,['FREQ ', num2str(F)]); %set frequency to Hz
%--------------------------------------

shipx=shipmodernization(shipx,perx,eu(1));
end
end




if 0%abs(eu(3))>10
putvalue(do.line(MX),OFF);
putvalue(do.line(MY),ON);
putvalue(do.line(MZ),OFF);
[u_tempy, pery]=fren(eu(3),shipy);
% display(eu(3));
display(u_tempy);

if abs(u_tempy)>u_max
u(k)=sign(u_tempy)*u_max;
else
u(k)=u_tempy;
end
%--- Send control effort to system ---
if u(k)>=0
putvalue(do.line(DIRY),0);  % Down --
else
putvalue(do.line(DIRY),1);  % Up --
end
F=abs(u(k));
fprintf(obj2,['FREQ ', num2str(F)]); %set frequency to Hz
%--------------------------------------

shipy=shipmodernization(shipy,pery,eu(3));
end


if 0%abs(eu(2))>10
putvalue(do.line(MX),OFF);
putvalue(do.line(MY),OFF);
putvalue(do.line(MZ),ON);
[u_tempz, perz]=fren(eu(2),shipx);
% display(eu(2));
display(u_tempz);

if abs(u_tempz)>u_max
u(k)=sign(u_tempz)*u_max;
else
u(k)=u_tempz;
end


%--- Send control effort to system ---
if u(k)>=0
putvalue(do.line(DIRZ),RIGHT);  % Down --
else
putvalue(do.line(DIRZ),LEFT);  % Up --
end
F=abs(u(k));
fprintf(obj2,['FREQ ', num2str(F)]); %set frequency to Hz
%--------------------------------------
shipz=shipmodernization(shipz,perz,eu(2));
end
Ts(k) = toc(tstart);
R_Time(k+1)=R_Time(k)+Ts(k);


if sum(abs(eu))==0
putvalue(do.line(MX),OFF);
putvalue(do.line(MY),OFF);
putvalue(do.line(MZ),OFF);
end
toc
tic
end



%% fin

for l=1:1
delete (aiv);

F_END=0;
fprintf(obj2,['FREQ ', num2str(F_END)]); %set frequency to Hz

%------Turn OFF All Instruments and motor --------

putvalue(do.line(MX),OFF);
putvalue(do.line(MY),OFF);
putvalue(do.line(MZ),OFF);
delete(do);

fprintf(obj2,'OUTPUT OFF');
fclose(obj2);
% 
% y_final=xd(k+1);
mxNiDeleteContext(KinectHandles)
end
\end{lstlisting}